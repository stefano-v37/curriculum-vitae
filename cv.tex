% !TEX encoding = UTF-8
% !TEX program = pdflatex
% !TEX spellcheck = en_GB

\documentclass[english,a4paper, nologo, notitle]{myeuropasscv}
\usepackage[english]{babel}
\usepackage{enumitem}

\ecvname{Stefano Vanin}
\ecvaddress{Via Fabio Severo 40, 34126 Trieste, Trieste, Italy \\ 
{\includegraphics[width=0.4cm]{address_europass_icon.pdf}}\hspace{0.2mm} {\fontseries{m}\selectfont Via Cristo Re 34/A, 30016 Jesolo, Venice, Italy}}
\ecvmobile{+39 3469479467}
\ecvemail{s.vanin@outlook.it}
\ecvlinkedinpage{https://www.linkedin.com/in/stefano-vanin/}
\ecvdateofbirth{1995.06.03}
\ecvnationality{Italian}
\ecvgithubpage{https://github.com/stefano-v37}


\ecvpictureleft[height=5cm]{picture/me.jpg}

\begin{document}
  \begin{europasscv}

  \ecvpersonalinfo
  
      \ecvsection{Professional Experience}

  \ecvtitle{Wärtsilä Italia S.p.A.}{Systems Software Developer}
  \ecvitem{2020.09 -- present \\ San Dorligo della Valle \\ (Trieste, Italy)}{Programming and integration of a collaborative robot in production lines.
  
  The asset is a KUKA KMR iiwa, which is a 7 degree of freedom manipulator installed on top of an autonomous guided vehicle.
  }
  % The main project I am working on is the automation of the kitting of some sub-assemblies of cylinder heads using a collaborative paradigm to achieve manufacturing excellence goals, in particular mapping the status of the process inside other manufactory softwares and improving working conditions.
  
  \ecvitem{Tasks}{Design of the processes and layouts, programming of the cobot and design of tools to be produced in additive manufacturing for project-related needs.
  
  Less frequently, I am also working on integration of devices considering digital and analog signals, power supply and fieldbus communication.
  
  Relevant trainings:
  \begin{enumerate}[noitemsep, topsep=0pt]
    \item Java (codewithmosh.com, Udemy);
    \item KUKA Sunrise.OS programming 1 and 2 (KUKA Roboter, Grugliasco, Torino);
    \item Siemens NX (CPI Eng, Internal).
  \end{enumerate}}
  
  \ecvtitle{Wärtsilä Italia S.p.A.}{Marine Engineering Trainee}
  \ecvitem{2019.08 -- 2020.07 \\ San Dorligo della Valle \\ (Trieste, Italy)}{
    Development and application of data driven methods for the concept development of vessels' engine rooms.}
  
	\ecvitem{Tasks}{Design of engine room arrangements focused on 2 and 4-strokes marine engines, gearboxes, power
    transformers, shaft generators, fuel cells and energy storage systems.
    
    Preparation of appealing value propositions of the solution, tailored on the most important
    	indicators for the specific customer.
      
      Relevant projects:
      \begin{enumerate}[noitemsep, topsep=0pt]
        \item Studies about merchant and passenger vessels (business cases);
        \item Collaboration in METRO project (business case);
        \item Solar panels simulation (algorithm);
        \item Port and Anchorage Identification (algorithm);
      \end{enumerate}}
  
  \ecvsection{Studies}
  
  \ecvtitlelevel{Master Degree}{Mechanical Engineering}{LM-33}
  \ecvitem{\textcolor{ecvsectioncolor}{Università degli Studi di Trieste}
  
2017.09 -- 2020.04}{\mbox{Thesis: "Data driven design of low emission machinery configurations for bulk carriers compliant} with IMO 2050 Strategy"

  	Relator: Prof. Rodolfo Taccani, Professor of Fluid Machines at Università degli Studi di Trieste

Correlator: Andrea Zotti, Wärtsilä Marine Engineering Director

110/110}
  \newpage
  \ecvtitlelevel{Bachelor Degree}{Industrial Engineering}{L-9}
  \ecvitem{\textcolor{ecvsectioncolor}{Università degli Studi di Trieste}
  
2014.09 -- 2017.10}{Thesis: "Potenziamento umano tramite muscoli artificiali pneumatici"

English translation: "Human augmentation using Pneumatic Artificial Muscles (PAM)"

Relator: Prof. Sabrina Pricl,

Professor of Principles of Chemical Engineering at Università degli Studi di Trieste

104/110}
  
  \ecvtitle{High School}{Liceo Scientifico}
  \ecvitem{\textcolor{ecvsectioncolor}{Liceo Scientifico Galileo Galilei}
  	
  	San Donà di Piave (Venice, Italia)
  
2014.06}{Thesis: "Approfondimento sui fenomeni che interessano una moto da corsa durante un giro di pista"

English translation: "Analysis of the phenomena to which a motorbike is subjected during a racing lap"

85/100}

    \ecvsection{Other working experiences}

\ecvitem{\textcolor{ecvsectioncolor}{Beach services}
	
	\textcolor{ecvsectioncolor}{CO.GE.AR.}

Summers from 2012 to 2016

Jesolo (Venice, Italy)
}{Welcome customers, cleaning and maintenance of the beach facilities.}


\ecvitem{\textcolor{ecvsectioncolor}{Assembly of doors and windows}
	
	\textcolor{ecvsectioncolor}{Tagliapietra Serramenti}
	
	2011.06 -- 2011.08
	
	Jesolo (Venice, Italy)
}{Quality control of the entering materials, cleaning and reorder of the warehouse, assembly
of simple components, supporting colleagues on more complex asssemblies and externally
during on-site assemblies.}

  
  \ecvsection{Skills}
  
  \ecvblueitem{Technical skills}{
  % During my trainee I have been having the opportunity to improve one of my interest which is
  % that of coding. I have been daily writing procedural instruction using Python under Jupyter environment,
  % and developing functional and object oriented programming features on repositories
  % under GIT on Atlassian Bitbucket. I am having the luck to collaborate with software architects
  % that are deploying on AWS the \emph{Berth and anchorage identification} logic I have written and therefore
  % observing their high-level methodologies. Also, I have acquired some experience in the maritime industry, in particular on the field of the
  % power supply and distribution.
  	
  My skills deriving from Universitary studies relate to the sizing of mechanical devices, plants and piping with a
  focus on optimization. I have made several projects using MATLAB, EES,
  ModeFRONTIER, ANSYS CFX, ANSYS Fluent, SolidWorks and Simulation. Each of these
  projects required a report of the results and therefore I have some experience with the markups
  of \LaTeX{} and TiKz.

  At the moment I feel more comfortable using Siemens CAD environment.
  
  Lately, I have been practicing programming and for the moment I have my routine consisting of Python managed using Conda environment and Java built using Maven. The versioning of my projects is done using GIT, hosted on GitHub or Bitbucket.}
  
  \ecvblueitem{Soft skills}{
  I am creative and curious, I like to learn new things and I appreciate being in a technologically
  driven context where expanding the knowledge is a shared target.
  
  In my limited work experience, I found out that I enjoy relating with colleagues and customers
  and I am happy to work in a social stimulating group where I am keen to listen and partecipate
  when something smart crosses my mind.
  
  I am optimistic but at the same time realistic, recognizing where there is a limit and committing
  to reach it.  
  
  I love sport, in particular basketball and racing. I would say that playing basketball during high school helped make me an hard
  worker who strives to achieve results.}

  \ecvblueitem{Languages}{Mother tongue Italian, fluent English}
  
  \ecvblueitem{Driving licenses}{A2, B.}
  
  \vfill
  \ecvsection{Information Privacy}
  
  \ecvitem{}{
  “In compliance with the GDPR and Italian Legislative Decree no. 196 dated 30/06/2003, I
  hereby authorize the recipient of this document to use and process my personal details for the
  purpose of recruiting and selecting staff and I confirm to be informed of my rights in accordance
  to art. 7 of the above mentioned Decree”.}
  
  \end{europasscv}

\end{document}